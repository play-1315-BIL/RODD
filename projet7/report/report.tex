\documentclass[12pt]{article}

% essential packages
\usepackage[utf8]{inputenc}
\usepackage[french]{babel}
\usepackage[T1]{fontenc}

% other packages
\usepackage{packages}

\title{TP RODD}
\date{\today}
\author{\textsc{BRUNOD-INDRIGO Luca}, \textsc{BATY Léo}}

\begin{document}

\maketitle

\section{Construction du graphe}

\noindent Soit $p\in \mathcal{P}$. Définissons le graphe $G^p = (V^p, A^p)$ :
\begin{arrowlist}
    \item \textbf{Sommets} : $V^p = \{ (t, l, a, j)\in \llb 1, T \rrb \times \llb 1, l_{max} \rrb \times \llb 0, a_{max} \rrb \times (\mathcal{C} \cup \{0\}) | j=0 \implies a=0 \}\bigcup\, \{u^p\}$
    $u^p = (0, l^p, 0, 0)$ état initial de la parcelle $p$.
    \item \textbf{Arcs} : $A^p$ contient quatre types d'arcs différents, correspondant aux transitions entre les périodes $t\in \llb 1, T \rrb$ :
    \begin{bulletlist}
        \item jachère $\rightarrow$ jachère : $(t-1, l, 0, 0) \rightarrow (t, \min(l+1, l_{max}), 0, 0)$
        \item culture $\rightarrow$ jachère : $(t-1, l, a, j) \rightarrow (t, 1, 0, 0)$
        \item jachère $\rightarrow$ culture : $(t-1, l, 0, 0) \rightarrow (t, l, 1, j)$
        \item culture $\rightarrow$ culture : $(t-1, l, a, j) \rightarrow (t, l, a+1, j')$
    \end{bulletlist}
    \item \textbf{Poids} : $w^p : A^p \rightarrow \bbR$, tel que : 
    
    $\forall(u, v)\in A^p,\, u = (t, l, a, j), v = (t', l', a', j'),\, \boxed{w^p_{u, v} = R_p(l, a', j, j')}$
    \item Soient $j\in\mathcal{C}, t\in \llb 1, T\rrb$.
    
    On note $A^p(j, t)$ le sous-ensemble d'arcs de $A^p$ de la forme $(t-1, l, a, i) \rightarrow (t, l', a', j)$
\end{arrowlist}

\noindent Une rotation sur la parcelle $p$ correspond à un chemin dans $G^p$ de $u^p$ à un sommet de type $(T, l, a, j)$.

\section{Modélisation par un PLNE}

% On relie tous les sommets $t=T$ à $u^p$
\noindent Considérons les graphes $G^p,\, p\in \mathcal{P}$ définis dans la section précédente. On peut modéliser le problème de planification sous la forme d'un programme linéaire en nombres entiers :
\begin{arrowlist}
    \item \textbf{Variables de décision} : $\forall p\in \mathcal{P},\, \forall (u, v)\in A^p,\, x_{u, v}^p = \1_{\{(u, v)\text{ est dans la rotation de la parcelle }p \}}$
    \item \textbf{Objectif} : $\sum\limits_{p\in \mathcal{P}} \sum\limits_{v\in \delta^+(u^p)}x_{u^p, v}^p$
    \item \textbf{Contraintes} :
    \begin{bulletlist}
        \item Demande satisfaite à chaque période pour chaque culture :
        
        $$\sum_{p\in\mathcal{P}}\sum_{(u, v)\in A^p(j, t)} w^p_{u, v} x_{u, v}^p \geq D_{j, t},\, \forall t\in \llb 1, T \rrb,\, \forall j\in C$$
        \item Contraintes de type \textit{flot} le long des chemins/rotations :
        
        $$\sum_{v\in \delta^-(u)} x_{v, u}^p = \sum_{v\in \delta^+(u)} x_{u, v}^p,\, \forall p\in \mathcal{P},\, \forall u\in V^p$$
        \item Au plus une rotation par parcelle :
        
        $$ \sum_{v\in \delta^+(u^p)}x_{u^p, v}^p \leq 1,\, \forall p\in \mathcal{P}$$
    \end{bulletlist}
\end{arrowlist}

\noindent PLNE final :
\begin{minie}|s|[2] % minie = minimize
    {x}  % optimization variables
    {\sum_{p\in \mathcal{P}} \sum_{v\in \delta^+(u^p)}x_{u^p, v}^p} % objective function and label
    {} % label for optimization problem
    {} % optimization result
    \addConstraint{\sum_{v\in \delta^-(u)} x_{v, u}^p}{ = \sum_{v\in \delta^+(u)} x_{u, v}^p}{\forall p\in \mathcal{P},\, \forall u\in V^p}
    \addConstraint{\sum_{p\in\mathcal{P}}\sum_{(u, v)\in A^p(j, t)} w^p_{u, v} x_{u, v}^p}{\geq D_{j, t}\quad}{\forall t\in \llb 1, T \rrb,\, \forall j\in C}
    \addConstraint{\sum_{v\in \delta^+(u^p)}x_{u^p, v}^p}{\leq 1}{\forall p\in \mathcal{P}}
    \addConstraint{x_{u, v}^p}{\in \{0, 1\}}{\forall p\in \mathcal{P},\, \forall (u, v)\in A^p}
\end{minie}

\section{Résolution de l'instance}

\noindent Nous avons implémenté le PLNE décrit dans la section précédente en \textsc{Julia} interfacé avec le solveur \textsc{CPLEX}, afin de résoudre l'instance de l'énoncé. On obtient le résultat suivant :
\begin{bulletlist}
    \item Nombres de noeuds développés dans l'arbre de recherche : $0$
    \item Nombre de parcelle cultivées (i.e. valeur de l'objectif à l'optimum) : $\boxed{19}$
\end{bulletlist}

\section{Formulation \textit{étendue}}

\noindent Notons $\mathcal{R}$ l'ensemble des rotations possibles. Soient $r\in \mathcal{R},\, j\in \mathcal{C},\, t\in \llb 1, T \rrb$. On note $r(j, t)$ l'ensemble des arcs de la rotation $r$

\begin{minie}|s|[2] % minie = minimize
    {x}  % optimization variables
    {\sum_{r\in \mathcal{R}} x_r} % objective function and label
    {} % label for optimization problem
    {} % optimization result
    \addConstraint{\sum_{r\in \mathcal{R}} \sum_{(u, v)\in r(j, t)} w_{u, v} x_r}{\geq D_{j, t}}{\forall j\in \mathcal{C},\, \forall t\in \llb 1, T \rrb}
    \addConstraint{x_r}{\in \bbN}{\forall r\in \mathcal{R}}
\end{minie}

\todo[inline]{Que se passe-t-il si le rendement dépend de la parcelle $p$ considérée ? Peut-on écrire qqchose ?}

\section{Approche de génération de colonnes}

\todo[inline]{à rédiger}

\noindent Sous-problème :

$$\max_r \sum_{j\in \mathcal{C}}\sum_{t = 1}^T \left(\sum_{(u, v) \in r(j, t)} w_{u, v}\right) \mu_{j, t} $$

\end{document}