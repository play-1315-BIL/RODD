\documentclass[12pt]{article}

% essential packages
\usepackage[utf8]{inputenc}
\usepackage[french]{babel}
\usepackage[T1]{fontenc}

% other packages
\usepackage{packages}

\usepackage{tikz}
\usetikzlibrary{shapes.geometric, arrows, positioning, arrows.meta, calc, external, automata}
\tikzexternalize
\tikzexternaldisable
\makeatletter
\makeatother
\tikzstyle{cercle} = [circle, minimum height=1.4cm, text centered, draw=black, fill=forestgreen!50]
\tikzstyle{arrow} = [thick, ->, >=stealth, line width = .7mm, color=black!80]
\tikzstyle{fractional-arrow} = [thick, ->, >=stealth, line width = .1mm, double, double distance=0.5mm]

\definecolor{forestgreen}{rgb}{0.0, 0.27, 0.13}

\title{Projet 2 -  RODD}
\date{27 janvier 2021}
\author{\textsc{BATY Léo}, \textsc{BRUNOD-INDRIGO Luca}}

\begin{document}

\maketitle

\section{Modélisation du problème}

\begin{arrowlist}
    \item \textbf{Données} :
    \begin{bulletlist}
        \item $d_{ij}$ : distance entre la parcelle $i$ et la parcelle $j$
        \item $p_i$ : coût de sélection de la parcelle $i$
        \item $a_i$ : aire de la parcelle $i$
        \item $A_{min}$ : aire minimum de parcelles à sélectionner
        \item $A_{max}$ : aire maximum de parcelles à sélectionner
        \item $B$ : budget maximum
    \end{bulletlist}
    \item \textbf{Variables de décision} :
    \begin{bulletlist}
        \item $x_i = \1_{\{ \text{la parcelle $i$ est sélectionnée} \}}$
        \item $y_{ij} = \1_{\{\text{la parcelle $i$ et selectionnée, et la parcelle $j$ est la plus proche parcelle sélectionnée de la parcelle $i$} \}}$
    \end{bulletlist}
    \item Programme d’optimisation combinatoire fractionnaire linéaire associé :
    \begin{align*}
        \min_{x, y} &\quad\frac{\sum_{i \in S} \sum_{j \neq i} d_{ij} y_{ij}}{\sum_{i \in S} x_i} &\\
        s.t.   &\quad \sum_{i \in S} p_i x_i \leq B\\
               &\quad \sum_{i \in S} a_i x_i \geq A_{min}\\
               &\quad \sum_{i \in S} a_i x_i \leq A_{max}\\
               &\quad \sum_{j \neq i} y_{ij} = x_i   & \forall i \in S\\
               &\quad y_{ij} \leq x_j                & \forall i \in S \; \forall j \neq i\\
               &\quad x_i \in \{0, 1\}   & \forall i \in S\\
               &\quad y_{ij} \in \{0, 1\}   & \forall i \in S \; \forall j \neq i
    \end{align*}
\end{arrowlist}

\section{Implémentation et résultats}

\noindent Afin de résoudre le programme d'optimisation combinatoire fractionnaire linéaire, nous avons implémenté l'algorithme de Dinkelbach présenté dans l'énoncé en Julia.

\medskip

\noindent Voici les résulats obtenus pour les 3 instances de l'énoncé, en utilisant CPLEX pour résoudre les PLNE :

\begin{arrowlist}
    \item $A_{min} = 30, A_{max} = 35, B = 920$ :
    \begin{bulletlist}
        \item temps de calcul : $0.449146576$ s
        \item noeuds développés dans l'arbre de recherche : 0
        \item nombre d'itérations de l'algorithme : 1
        \item valeur de la solution : $1.1550093846624292$
        \item parcelles retenues : 35
        \begin{center}
            \begin{tikzpicture}[scale=0.5, every node/.style={transform shape}]
                \tikzstyle{box} = [rectangle,draw=black,thick, minimum size=1cm]
            
                \foreach \x in {1,...,10}{
                    \foreach \y in {1,...,10}
                        \node[box] at (\x,\y){};
                }
                
                \node[box, fill=forestgreen!50] at (2,10){};
                \node[box, fill=forestgreen!50] at (5,10){};
                \node[box, fill=forestgreen!50] at (8,10){};
                \node[box, fill=forestgreen!50] at (5,9){};
                \node[box, fill=forestgreen!50] at (8,9){};
                \node[box, fill=forestgreen!50] at (2,8){};
                \node[box, fill=forestgreen!50] at (3,8){};
                \node[box, fill=forestgreen!50] at (5,8){};
                \node[box, fill=forestgreen!50] at (6,8){};
                \node[box, fill=forestgreen!50] at (7,8){};
                \node[box, fill=forestgreen!50] at (2,7){};
                \node[box, fill=forestgreen!50] at (5,7){};
                \node[box, fill=forestgreen!50] at (1,6){};
                \node[box, fill=forestgreen!50] at (2,6){};
                \node[box, fill=forestgreen!50] at (3,6){};
                \node[box, fill=forestgreen!50] at (4,6){};
                \node[box, fill=forestgreen!50] at (7,6){};
                \node[box, fill=forestgreen!50] at (10,6){};
                \node[box, fill=forestgreen!50] at (3,5){};
                \node[box, fill=forestgreen!50] at (2,4){};
                \node[box, fill=forestgreen!50] at (3,4){};
                \node[box, fill=forestgreen!50] at (10,4){};
                \node[box, fill=forestgreen!50] at (2,3){};
                \node[box, fill=forestgreen!50] at (4,3){};
                \node[box, fill=forestgreen!50] at (5,3){};
                \node[box, fill=forestgreen!50] at (9,3){};
                \node[box, fill=forestgreen!50] at (10,3){};
                \node[box, fill=forestgreen!50] at (4,2){};
                \node[box, fill=forestgreen!50] at (5,1){};
                \node[box, fill=forestgreen!50] at (8,1){};
            
            \end{tikzpicture}
        \end{center}
    \end{bulletlist}
    \item $A_{min} = 20, A_{max} = 21, B = 520$ :
    \begin{bulletlist}
        \item temps de calcul : $0.318542181$ s
        \item noeuds développés dans l'arbre de recherche : 0
        \item nombre d'itérations de l'algorithme : 1
        \item valeur de la solution : $1.2739354332309538$
        \item parcelles retenues : 20
        \begin{center}
            \begin{tikzpicture}[scale=0.5, every node/.style={transform shape}]
                \tikzstyle{box} = [rectangle,draw=black,thick, minimum size=1cm]
            
                \foreach \x in {1,...,10}{
                    \foreach \y in {1,...,10}
                        \node[box] at (\x,\y){};
                }
                
                \node[box, fill=forestgreen!50] at (2,10){};
                \node[box, fill=forestgreen!50] at (5,10){};
                \node[box, fill=forestgreen!50] at (8,10){};

                \node[box, fill=forestgreen!50] at (5,9){};
                \node[box, fill=forestgreen!50] at (8,9){};

                \node[box, fill=forestgreen!50] at (2,8){};
                \node[box, fill=forestgreen!50] at (5,8){};
                \node[box, fill=forestgreen!50] at (6,8){};

                \node[box, fill=forestgreen!50] at (2,7){};

                \node[box, fill=forestgreen!50] at (1,6){};
                \node[box, fill=forestgreen!50] at (3,6){};

                \node[box, fill=forestgreen!50] at (3,5){};

                \node[box, fill=forestgreen!50] at (2,4){};
                \node[box, fill=forestgreen!50] at (3,4){};
                \node[box, fill=forestgreen!50] at (10,4){};

                \node[box, fill=forestgreen!50] at (2,3){};
                \node[box, fill=forestgreen!50] at (5,3){};
                \node[box, fill=forestgreen!50] at (9,3){};

                \node[box, fill=forestgreen!50] at (5,1){};
                \node[box, fill=forestgreen!50] at (8,1){};
            
            \end{tikzpicture}
        \end{center}
    \end{bulletlist}
    \item $A_{min} = 70, A_{max} = 75, B = 3500$ :
    \begin{bulletlist}
        \item temps de calcul : $1.17581296$ s
        \item noeuds développés dans l'arbre de recherche : 0
        \item nombre d'itérations de l'algorithme : 1
        \item valeur de la solution : $1$
        \item parcelles retenues : 70
        \begin{center}
            \begin{tikzpicture}[scale=0.5, every node/.style={transform shape}]
                \tikzstyle{box} = [rectangle,draw=black,thick, minimum size=1cm]
            
                \foreach \x in {1,...,10}{
                    \foreach \y in {1,...,10}
                        \node[box, fill=forestgreen!50] at (\x,\y){};
                }
                
                \node[box, fill=white] at (3,10){};
                \node[box, fill=white] at (4,10){};

                \node[box, fill=white] at (3,9){};
                \node[box, fill=white] at (6,9){};
                \node[box, fill=white] at (9,9){};
                \node[box, fill=white] at (10,9){};

                \node[box, fill=white] at (8,8){};
                \node[box, fill=white] at (9,8){};

                \node[box, fill=white] at (2,7){};
                \node[box, fill=white] at (8,7){};
                \node[box, fill=white] at (9,7){};

                \node[box, fill=white] at (5,6){};
                \node[box, fill=white] at (8,6){};
                \node[box, fill=white] at (9,6){};

                \node[box, fill=white] at (4,5){};
                \node[box, fill=white] at (5,5){};
                \node[box, fill=white] at (6,5){};
                \node[box, fill=white] at (9,5){};
                \node[box, fill=white] at (10,5){};

                \node[box, fill=white] at (5,4){};
                \node[box, fill=white] at (6,4){};
                \node[box, fill=white] at (8,4){};

                \node[box, fill=white] at (3,3){};
                \node[box, fill=white] at (8,3){};

                \node[box, fill=white] at (1,2){};
                \node[box, fill=white] at (7,2){};

                \node[box, fill=white] at (1,1){};
                \node[box, fill=white] at (7,1){};
                \node[box, fill=white] at (9,1){};
                \node[box, fill=white] at (10,1){};
            
            \end{tikzpicture}
        \end{center}
    \end{bulletlist}
\end{arrowlist}

\end{document}