\documentclass[12pt]{extarticle}
\usepackage[francais]{babel}
\usepackage[utf8]{inputenc}
\usepackage[T1]{fontenc}
\usepackage{graphicx}
\usepackage{framed}
\usepackage[normalem]{ulem}
\usepackage{amsmath}
\usepackage{amsthm}
\usepackage{amssymb}
\usepackage{amsfonts}
\usepackage{enumerate}
\usepackage[top=1 in,bottom=1in, left=1 in, right=1 in]{geometry}
\usepackage{subcaption}
\usepackage[usenames,dvipsnames]{xcolor}
\usepackage{packages}
\usepackage{multirow}

\title{Projet 3 -  RODD}
\date{3 février 2021}
\author{\textsc{BATY Léo}, \textsc{BRUNOD-INDRIGO Luca}}

\begin{document}

\maketitle

\section{Premier modèle}

Notons $\mathcal{F}$ (resp. $\mathcal{M}$) l'ensemble des individus femelles (resp. mâles) et $\mathcal{K}^1_{ij}$ (resp. $\mathcal{K}^2_{ij}$) l'ensemble des individus ayant un (resp. deux) exemplaire(s) de l'allèle $j$ du gène $i$.\\
On considère également les variables entières $(x_k)_{k=1 \dots p}$ donnant le nombre de progénitures de l'individu $k$.

\bigskip

L'espérance du nombre d'allèles disparus s'écrit : 

\begin{align*}
    \mathbb{E}(N^{disparus})    &= \sum_{i \leq M} \sum_{j \leq t(i)} \mathbb{E}(\1_{ij \; \text{disparait}}) \\
                                &= \sum_{i \leq M} \sum_{j \leq t(i)} \mathbb{P}(ij \; \text{disparait})
\end{align*}

Or, on remarque que : 
$$
\mathbb{P}(ij \; \text{disparait}) = \left\{
    \begin{array}{ll}
        0 & \mbox{si } \exists k \in \mathcal{K}^2_{ij} \; \mbox{tel que} \;  x_k > 0 \\
        \left(\frac{1}{2}\right)^{\sum_{k \in K_{ij}^1} x_k} & \mbox{sinon.}
    \end{array}
\right.
$$

Ce qui, les $x_k$ étant entiers, se réécrit : 

$$\mathbb{P}(ij \; \text{disparait}) = \max(0, \left(\frac{1}{2}\right)^{\sum_{k \in K_{ij}^1} x_k} - \sum_{k \in K_{ij}^2} x_k)$$

On en déduit le modèle suivant : 

\begin{align*}
    \min_{x, p, t} &\sum_{i \leq M} \sum_{j \leq t(i)} p_{ij}\\
    s.t.   & \log(t_{ij}) \geq \sum_{k \in K_{ij}^1} x_k \log\left(\frac{1}{2}\right)\\
           & p_{ij} \geq t_{ij} - \sum_{k \in K_{ij}^2} x_k\\
           & \sum_{k \in \mathcal{M}} x_k = P\\
           & \sum_{k \in \mathcal{F}} x_k = P\\
           & x_k \in \mathbb{N}\\
           & p_{ij} \geq 0
\end{align*}

\section{Approximation linéaire}

En approchant la fonction logarithme par une fonction affine par morceaux comme indiqué dans le sujet, on obtient le modèle suivant :

\begin{align*}
    \min_{x, p} &\sum_{i \leq M} \sum_{j \leq t(i)} p_{ij}\\
    s.t.   & \log(\theta_r) + \frac{1}{\theta_r}(t_{ij} - \theta_r) \geq \sum_{k \in K_{ij}^1} x_k \log\left(\frac{1}{2}\right)\\
           & p_{ij} \geq t_{ij} - \sum_{k \in K_{ij}^2} x_k     \\      
           & \sum_{k \in \mathcal{M}} x_k = P\\
           & \sum_{k \in \mathcal{F}} x_k = P\\
           & x_k \in \mathbb{N}\\
           & p_{ij} \geq 0
\end{align*}

Ce modèle fournit bien une solution admissible au problème, étant donné que le vecteur $x_k$ vérifie toujours les contraintes d'intégrité et de conservation de la taille de la population.\\
De plus, puisqu'il s'agit d'une approximation par excès du logarithme, les contraintes $\log(\theta_r) + \frac{1}{\theta_r}(t_{ij} - \theta_r) \geq \sum_{k \in K_{ij}^1} x_k \log\left(\frac{1}{2}\right)$ à $x_{k}$ fixés conduisent à des valeurs de $t_{ij}$ plus faibles que la contrainte $\log(t_{ij}) \geq \sum_{k \in K_{ij}^1} x_k \log\left(\frac{1}{2}\right)$, donc également des valeurs plus faibles des $p_{ij}$ et de l'objectif.\\
Une solution optimale de ce modèle est donc bien une borne inférieure.

\section{Calculs sur les instances de l'énoncé}

Le tableau \ref{tab:table-1} montre les résultats obtenus sur les instances de l'énoncé en prenant $\theta_1 = 0.0001$ et $h = 50$.

\begin{table}[h!]
    \scriptsize
    \centering
    \begin{tabular}{|c|c|c|c|c|c|c|c|c|c|c|c|c|c|c|}
        \hline
        \multirow{2}{*}{\textbf{Instance}} & \multirow{2}{*}{\textbf{Temps}} & \multirow{2}{*}{\textbf{Nombre de noeuds}} & \multirow{2}{*}{\textbf{Espérance}} & \multicolumn{10}{|c|}{\textbf{Probabilités de disparition}} & \textbf{Borne inférieure} \\
        & & & & \textbf{A} & \textbf{a} & \textbf{B} & \textbf{b} & \textbf{C} & \textbf{c} & \textbf{D} & \textbf{d} & \textbf{E} & \textbf{e} & \\
        \hline
        \textbf{$x_k \leq 2$} & 0.00 & 0 & 0.0625 & 0 & 0 & 0 & 0.0625 & 0 & 0 & 0 & 0 & 0 & 0 & 0.06243336502896749 \\
        \hline
        \textbf{$x_k \leq 3$} & 0.15 & 0 & 0.015625 & 0 & 0 & 0 & 0.015625 & 0 & 0 & 0 & 0 & 0 & 0 & 0.015620569018547845 \\
        \hline
    \end{tabular}
    \caption{Résultats sur les instances de l'énoncé}
    \label{tab:table-1}
\end{table}

\section{Essais sur des instances aléatoires}

Le tableau \ref{tab:table-2} donne les résultats obtenus en appliquent le programme à des instances de tailles différentes en conservant les paramètres $\theta_1 = 0.0001$ et $h = 50$.

\begin{table}[h!]
    \scriptsize
    \centering
    \begin{tabular}{|c|c|c|c|c|c|}
        \hline
        \textbf{Taille de population} & \textbf{Nombre de gènes} &\textbf{Temps} & \textbf{Nombre de noeuds} & \textbf{Espérance} &  \textbf{Borne inférieure} \\
        \hline
        10 & 6 & 0.03 & 0 & 3.814697265625e-6 & 0.0 \\
        \hline
        12 & 8 & 0.02 & 0 & 0 & 0.0 \\
        \hline
        16 & 10 & 0.03 & 0 & 6.103562191128731e-5 & 0.0 \\
        \hline
        20 & 16 & 0.01 & 0 & 6.198883056640625e-6 & 0.0 \\
        \hline
        30 & 19 & 0.02 & 0 & 5.122274165589302e-9 & 0.0 \\
        \hline
        40 & 25 & 0.04 & 0 & 5.911715561532305e-12 & 0.0 \\
        \hline
        50 & 31 & 0.06 & 0 & 7.549516567451072e-15 & 0.0 \\
        \hline
        60 & 38 & 0.09 & 0 & 4.662069341687669e-18 & 0.0 \\
        \hline
        80 & 50 & 0.19 & 0 & 9.155273437500011e-5 & 0.0 \\
        \hline
        100 & 62 & 0.27 & 0 & 6.103515625e-5 & 0.0 \\
        \hline
        160 & 100 & 0.80 & 0 & 0.00010681338608264923 & 0.0 \\
        \hline
        200 & 125 & 0.97 & 0 & 0.00018310640007257462 & 0.0 \\
        \hline
        300 & 188 & 2.84 & 0 & 7.629767065964188e-5 & 0.0 \\
        \hline
        400 & 250 & 17.49 & 0 & 6.104307252030594e-5 & 0.0 \\
        \hline
        500 & 312 & 14.50 & 0 & 0.00011062668636441231 & 0.0 \\
    \end{tabular}
    \caption{Résultats sur des instances aléatoires de tailles différentes}
    \label{tab:table-2}
\end{table}

On peut constater que, malgré l'augmentation de taille, CPLEX réussit toujours à résoudre le problème à la racine.
Par ailleurs, et comme cela pouvait être attendu, l'écart entre la borne inférieure et l'espérance calculée à partir de la solution obtenue semble augmenter.
Nous pensons que cela est dû au fait que le manque de précision dans l'approximation du logarithme a davantage d'impact quand la taille de l'instance augmente.

\begin{table}[h!]
    \scriptsize
    \centering
    \begin{tabular}{|c|c|c|c|c|c|}
        \hline
        \textbf{$h$} &\textbf{Temps} & \textbf{Nombre de noeuds} & \textbf{Espérance} &  \textbf{Borne inférieure} \\
        \hline
        50 & 1.1836340427398682 & 0 & 0.0001068115234375 & 0.0\\
        \hline
        60 & 1.3889548778533936 & 0 & 0.0001068115234375 & 0.0\\
        \hline
        70 & 1.713580846786499 & 0 & 0.0001068115234375 & 0.0\\
        \hline
        80 & 1.942518949508667 & 0 & 0.0001068115234375 & 0.0\\
        \hline
        90 & 2.1335439682006836 & 0 & 0.0001068115234375 & 0.0\\
        \hline
        100 & 2.448179006576538 & 0 & 0.0001068115234375 & 0.0\\
        \hline
        110 & 2.729564905166626 & 0 & 0.0001068115234375 & 0.0\\
        \hline
        120 & 3.1832730770111084 & 0 & 0.0001068115234375 & 0.0\\
        \hline
        130 & 3.364945888519287 & 0 & 0.0001068115234375 & 0.0\\
        \hline
        140 & 3.5438148975372314 & 0 & 0.0001068115234375 & 0.0\\
        \hline
        150 & 3.849707841873169 & 0 & 0.0001068115234375 & 0.0\\
        \hline
        160 & 4.177340030670166 & 0 & 0.0001068115234375 & 0.0\\
        \hline
        170 & 4.513378143310547 & 0 & 0.0001068115234375 & 0.0\\
        \hline
        180 & 5.585042953491211 & 0 & 6.866548210382462e-5 & 0.0\\
        \hline
        190 & 6.031890153884888 & 0 & 6.866548210382462e-5 & 0.0\\
        \hline
        200 & 5.396224021911621 & 0 & 0.0001068115234375 & 0.0\\
        \hline
    \end{tabular}
    \caption{Résultats sur une instance aléatoire pour différentes valeurs de $h$}
    \label{tab:table-3}
\end{table}

Le tableau \ref{tab:table-3} montre les résultats obtenus en appliquant le programme à une même instance aléatoire pour différentes valeurs de $h$.
On remarque naturellement que l'augmentation de $h$ augmente le temps de calcul puisqu'il y a davantage de contraintes.
Cependant, alors qu'on aurait pu s'attendre à ce que l'espérance obtenue augmente du fait de la meilleure qualité de l'approximation du logarithme, on observe seulement deux valeurs différentes qui ne permettent pas de définir une tendance.
Cela est pourrait être dû à une spécificité de l'instance aléatoire utilisée.



\end{document}